% ----------- Author: Radoslav Grenčík, xgrenc00@stud.fit.vutbr.cz ----------- %

\documentclass[a4paper, 11pt]{article}
\usepackage[slovak]{babel}
\usepackage[utf8]{inputenc}
\usepackage[left=2cm, top=3cm, text={17cm, 24cm}]{geometry}
\usepackage{times}
\usepackage{graphicx}
\usepackage{csquotes}
\MakeOuterQuote{"}
\usepackage[hyphens]{url}
\usepackage{hyperref}
\hypersetup{hidelinks}
\usepackage[perpage]{footmisc}
\usepackage{multirow}
\usepackage[normalem]{ulem}
\useunder{\uline}{\ul}{}

\begin{document}

% ---------------------------------------------------------------------------- %
%                                  TITLE PAGE                                  %
% ---------------------------------------------------------------------------- %
\begin{titlepage}
	\begin{center}
		\includegraphics[width=0.77 \linewidth]{FIT_logo.pdf}
		
		\vspace{\stretch{0.382}}
		
		\huge{IMS 2020/21 - Simulačná štúdia} \\
		\LARGE{\textbf{Téma č. 6: Modelování vodohospodářství}}
		
		\vspace{\stretch{0.618}}
	\end{center}
	\begin{minipage}{0.25 \textwidth}
		\begin{flushleft}
			\Large
			\today
		\end{flushleft}
	\end{minipage}
	\hfill
	\begin{minipage}{0.65 \textwidth}
		\begin{flushright}
			\Large
			Radoslav Grenčík, \\
			\large
			\texttt{\href{mailto:xgrenc00@stud.fit.vutbr.cz}{xgrenc00@stud.fit.vutbr.cz}}
		\end{flushright}
	\end{minipage}
\end{titlepage}

% ---------------------------------------------------------------------------- %
%                               TABLE OF CONTENTS                              %
% ---------------------------------------------------------------------------- %
\clearpage
\thispagestyle{empty}
\tableofcontents

% ---------------------------------------------------------------------------- %
% ---------------------------------------------------------------------------- %
% ---------------------------------------------------------------------------- %
\clearpage
\pagenumbering{arabic}
\setcounter{page}{3}

\section{Úvod}

Kvôli nedostupnosti a nedostatku informácií o téme modelovanie vodohospodárstva, som sa rozhodol nadviazať na projekt, ktorý som vypracoval spolu so študentom Robertom Hubinákom do predmetu IMS v akademickom roku 2019/2020 \cite{IMS_project}. Spomínaný projekt sa zameriava na problémy s extrémnym prebytkom plastového odpadu na našej planéte. Podľa môjho názoru je práve táto téma veľmi blízko previazaná so znečistením svetových oceánov tonami plastového odpadu. Práve preto je cieľom tejto práce poukázať na tento problém, vytvoriť model, ktorý popisuje túto kritickú situáciu a vyvodiť riešenia, ktoré by mohli zabrániť v ďalšom znečisťovaní našej vzácnej planéty Zeme.

\subsection{Autor, zdroje}

Projekt vypracoval študent VUT FIT v Brne Radoslav Grenčík.

K vypracovaniu projektu boli využité poznatky a študijné texty z predmetu Modelování a simulace \cite{IMS_slides}, ktorý sa vyučuje na VUT FIT v Brne. Ako zdroj údajov slúžili rôzne štúdie a články na internete. Projekt naväzuje na projekt, ktorý som vypracoval spolu so študentom Robertom Hubinákom do predmetu IMS v akademickom roku 2019/2020 \cite{IMS_project}. Touto cestou by som mu chcel poďakovať za povolenie použiť materiály, ktoré boli vypracované do spomínaného projektu.

\subsection{Overovanie validity modelu}

Validita modelu bola overovaná experimentovaním a porovnávaním výsledkov s reálnymi nameranými dátami, ktoré boli čerpané z overených zdrojov.

% ---------------------------------------------------------------------------- %
% ---------------------------------------------------------------------------- %
% ---------------------------------------------------------------------------- %
\pagebreak
\section{Rozbor témy a použitých metód/technológií}



\subsection{Použité postupy}

Pre vytvorenie simulačného modelu bol využitý programovací jazyk C++. Ďalej boli použité postupy popísané v prednáškach k predmetu Modelování a simulace \cite{IMS_slides}, ktorý je vyučovaný na VUT FIT v Brne.

\subsection{Popis pôvodu použitých metód a technológii}

Boli použité štandardné knižnice jazyka C++. Ako nástroj pre preklad bol použitý GNU Make \cite{make}.

% ---------------------------------------------------------------------------- %
% ---------------------------------------------------------------------------- %
% ---------------------------------------------------------------------------- %
\pagebreak
\section{Koncepcia metódy, prístupu, modelu}



\subsection{Popis konceptuálneho modelu}



\subsection{Forma konceptuálneho modelu}



% ---------------------------------------------------------------------------- %
% ---------------------------------------------------------------------------- %
% ---------------------------------------------------------------------------- %
\pagebreak
\section{Architektúra simulačného modelu/simulátoru}



\subsection{Mapovanie konceptuálneho modelu do simulačného modelu}



\subsection{Spustenie simulačného modelu, parametre programu}



\subsubsection{Popis parametrov programu}



% ---------------------------------------------------------------------------- %
% ---------------------------------------------------------------------------- %
% ---------------------------------------------------------------------------- %
\pagebreak
\section{Podstata simulačných experimentov a ich priebeh}



\subsection{Simulačné experimenty}

\subsubsection{Experiment 1}



\subsubsection{Experiment 2}



\subsubsection{Experiment 3}



\subsubsection{Experiment 4}



\subsubsection{Experiment 5}


% ---------------------------------------------------------------------------- %
% ---------------------------------------------------------------------------- %
% ---------------------------------------------------------------------------- %
\pagebreak
\section{Zhrnutie simulačných experimentov a záver}



% ---------------------------------------------------------------------------- %
%                                   CITATIONS                                  %
% ---------------------------------------------------------------------------- %
\clearpage
\bibliographystyle{czechiso}
\renewcommand{\refname}{Literatúra}
\bibliography{documentation}

\end{document}
